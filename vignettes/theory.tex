% Options for packages loaded elsewhere
\PassOptionsToPackage{unicode}{hyperref}
\PassOptionsToPackage{hyphens}{url}
%
\documentclass[
]{article}
\usepackage{lmodern}
\usepackage{amssymb,amsmath}
\usepackage{ifxetex,ifluatex}
\ifnum 0\ifxetex 1\fi\ifluatex 1\fi=0 % if pdftex
  \usepackage[T1]{fontenc}
  \usepackage[utf8]{inputenc}
  \usepackage{textcomp} % provide euro and other symbols
\else % if luatex or xetex
  \usepackage{unicode-math}
  \defaultfontfeatures{Scale=MatchLowercase}
  \defaultfontfeatures[\rmfamily]{Ligatures=TeX,Scale=1}
\fi
% Use upquote if available, for straight quotes in verbatim environments
\IfFileExists{upquote.sty}{\usepackage{upquote}}{}
\IfFileExists{microtype.sty}{% use microtype if available
  \usepackage[]{microtype}
  \UseMicrotypeSet[protrusion]{basicmath} % disable protrusion for tt fonts
}{}
\makeatletter
\@ifundefined{KOMAClassName}{% if non-KOMA class
  \IfFileExists{parskip.sty}{%
    \usepackage{parskip}
  }{% else
    \setlength{\parindent}{0pt}
    \setlength{\parskip}{6pt plus 2pt minus 1pt}}
}{% if KOMA class
  \KOMAoptions{parskip=half}}
\makeatother
\usepackage{xcolor}
\IfFileExists{xurl.sty}{\usepackage{xurl}}{} % add URL line breaks if available
\IfFileExists{bookmark.sty}{\usepackage{bookmark}}{\usepackage{hyperref}}
\hypersetup{
  pdftitle={The theory behind onlineFDR},
  hidelinks,
  pdfcreator={LaTeX via pandoc}}
\urlstyle{same} % disable monospaced font for URLs
\usepackage[margin=1in]{geometry}
\usepackage{color}
\usepackage{fancyvrb}
\newcommand{\VerbBar}{|}
\newcommand{\VERB}{\Verb[commandchars=\\\{\}]}
\DefineVerbatimEnvironment{Highlighting}{Verbatim}{commandchars=\\\{\}}
% Add ',fontsize=\small' for more characters per line
\usepackage{framed}
\definecolor{shadecolor}{RGB}{248,248,248}
\newenvironment{Shaded}{\begin{snugshade}}{\end{snugshade}}
\newcommand{\AlertTok}[1]{\textcolor[rgb]{0.94,0.16,0.16}{#1}}
\newcommand{\AnnotationTok}[1]{\textcolor[rgb]{0.56,0.35,0.01}{\textbf{\textit{#1}}}}
\newcommand{\AttributeTok}[1]{\textcolor[rgb]{0.77,0.63,0.00}{#1}}
\newcommand{\BaseNTok}[1]{\textcolor[rgb]{0.00,0.00,0.81}{#1}}
\newcommand{\BuiltInTok}[1]{#1}
\newcommand{\CharTok}[1]{\textcolor[rgb]{0.31,0.60,0.02}{#1}}
\newcommand{\CommentTok}[1]{\textcolor[rgb]{0.56,0.35,0.01}{\textit{#1}}}
\newcommand{\CommentVarTok}[1]{\textcolor[rgb]{0.56,0.35,0.01}{\textbf{\textit{#1}}}}
\newcommand{\ConstantTok}[1]{\textcolor[rgb]{0.00,0.00,0.00}{#1}}
\newcommand{\ControlFlowTok}[1]{\textcolor[rgb]{0.13,0.29,0.53}{\textbf{#1}}}
\newcommand{\DataTypeTok}[1]{\textcolor[rgb]{0.13,0.29,0.53}{#1}}
\newcommand{\DecValTok}[1]{\textcolor[rgb]{0.00,0.00,0.81}{#1}}
\newcommand{\DocumentationTok}[1]{\textcolor[rgb]{0.56,0.35,0.01}{\textbf{\textit{#1}}}}
\newcommand{\ErrorTok}[1]{\textcolor[rgb]{0.64,0.00,0.00}{\textbf{#1}}}
\newcommand{\ExtensionTok}[1]{#1}
\newcommand{\FloatTok}[1]{\textcolor[rgb]{0.00,0.00,0.81}{#1}}
\newcommand{\FunctionTok}[1]{\textcolor[rgb]{0.00,0.00,0.00}{#1}}
\newcommand{\ImportTok}[1]{#1}
\newcommand{\InformationTok}[1]{\textcolor[rgb]{0.56,0.35,0.01}{\textbf{\textit{#1}}}}
\newcommand{\KeywordTok}[1]{\textcolor[rgb]{0.13,0.29,0.53}{\textbf{#1}}}
\newcommand{\NormalTok}[1]{#1}
\newcommand{\OperatorTok}[1]{\textcolor[rgb]{0.81,0.36,0.00}{\textbf{#1}}}
\newcommand{\OtherTok}[1]{\textcolor[rgb]{0.56,0.35,0.01}{#1}}
\newcommand{\PreprocessorTok}[1]{\textcolor[rgb]{0.56,0.35,0.01}{\textit{#1}}}
\newcommand{\RegionMarkerTok}[1]{#1}
\newcommand{\SpecialCharTok}[1]{\textcolor[rgb]{0.00,0.00,0.00}{#1}}
\newcommand{\SpecialStringTok}[1]{\textcolor[rgb]{0.31,0.60,0.02}{#1}}
\newcommand{\StringTok}[1]{\textcolor[rgb]{0.31,0.60,0.02}{#1}}
\newcommand{\VariableTok}[1]{\textcolor[rgb]{0.00,0.00,0.00}{#1}}
\newcommand{\VerbatimStringTok}[1]{\textcolor[rgb]{0.31,0.60,0.02}{#1}}
\newcommand{\WarningTok}[1]{\textcolor[rgb]{0.56,0.35,0.01}{\textbf{\textit{#1}}}}
\usepackage{graphicx,grffile}
\makeatletter
\def\maxwidth{\ifdim\Gin@nat@width>\linewidth\linewidth\else\Gin@nat@width\fi}
\def\maxheight{\ifdim\Gin@nat@height>\textheight\textheight\else\Gin@nat@height\fi}
\makeatother
% Scale images if necessary, so that they will not overflow the page
% margins by default, and it is still possible to overwrite the defaults
% using explicit options in \includegraphics[width, height, ...]{}
\setkeys{Gin}{width=\maxwidth,height=\maxheight,keepaspectratio}
% Set default figure placement to htbp
\makeatletter
\def\fps@figure{htbp}
\makeatother
\setlength{\emergencystretch}{3em} % prevent overfull lines
\providecommand{\tightlist}{%
  \setlength{\itemsep}{0pt}\setlength{\parskip}{0pt}}
\setcounter{secnumdepth}{-\maxdimen} % remove section numbering

\title{The theory behind onlineFDR}
\author{}
\date{\vspace{-2.5em}}

\begin{document}
\maketitle

{
\setcounter{tocdepth}{2}
\tableofcontents
}
\begin{center}\rule{0.5\linewidth}{0.5pt}\end{center}

\hypertarget{theory-behind-onlinefdr}{%
\subsection{Theory behind onlineFDR}\label{theory-behind-onlinefdr}}

\hypertarget{online-hypothesis-testing}{%
\subsubsection{Online hypothesis
testing}\label{online-hypothesis-testing}}

Consider a sequence of hypotheses \(H_1, H_2, H_3, \ldots\) that arrive
sequentially in a stream, with corresponding \(p\)-values
\((p_1, p_2, p_3, \ldots)\). A testing procedure provides a sequence of
adjusted significance thresholds \(\alpha_i\), with corresponding
decision rule: \[ R_i = \begin{cases}
1 & \text{if } p_i \leq \alpha_i & (\text{reject } H_i)\\
0 & \text{otherwise} & (\text{accept } H_i)
\end{cases} \]

In \emph{online} testing, the significance thresholds can only be
functions of the prior decisions,
i.e.~\(\alpha_i = \alpha_i(R_1, R_2, \ldots, R_{i-1})\).

Javanmard and Montanari (2015, 2018) proposed two procedures for online
control. The first is LOND, which stands for (significance) Levels based
On Number of Discoveries. The second is LORD, which stands for
(significance) Levels based On Recent Discovery. LORD was subsequently
extended by Ramdas \emph{et al.} (2017). Ramdas \emph{et al.} (2018)
also proposed the SAFFRON procedure, which provides an adaptive method
of online FDR control, which includes a variant of Alpha-investing.
Finally, Tian \& Ramdas (2019) proposed the ADDIS procedure as an
improvement of SAFFRON in the presence of conservative nulls.

\hypertarget{LOND}{%
\subsubsection{LOND}\label{LOND}}

The LOND procedure controls the FDR for independent or positively
dependent (PRDS) \(p\)-values. Given an overall significance level
\(\alpha\), we choose a sequence of non-negative numbers
\(\beta = (\beta_i)_{i \in \mathbb{N}}\) such that they sum to
\(\alpha\). The values of the adjusted significance thresholds
\(\alpha_i\) are chosen as follows:
\[ \alpha_i = \beta_i (D(i-1) + 1) \] where \(D(n) = \sum_{i=1}^n R_i\)
denotes the number of discoveries (i.e.~rejections) in the first \(n\)
hypotheses tested.

LOND can be adjusted to also control FDR under arbitrarily dependent
\(p\)-values. To do so, it is modified with
\(\tilde{\beta}_i = \beta_i/H(i)\) in place of \(\beta_i\), where
\(H(i) = \sum_{j=1}^i \frac{1}{j}\) is the \(i\)-th harmonic number.
Note that this leads to a substantial loss in power compared to the
unadjusted LOND procedure. The correction factor is similar to the
classical one used by Benjamini and Yekutieli (2001), except that in
this case the \(i\)-th hypothesis among \(N\) is penalised by a factor
of \(H(i)\) to give consistent results across time (as compared to a
factor \(H(N)\) for the Benjamini and Yekutieli method).

The default sequence of \(\beta\) is given by
\[\beta_j = C \alpha \frac{\log(\max(j, 2))}{j e^{\sqrt{\log j}}}\]
where \(C \approx 0.07720838\), as proposed by Javanmard and Montanari
(2018) equation 31.

\hypertarget{LORD}{%
\subsubsection{LORD}\label{LORD}}

The LORD procedure controls the FDR for independent \(p\)-values. We
first fix a sequence of non-negative numbers
\(\gamma = (\gamma_i)_{i \in \mathbb{N}}\) such that
\(\gamma_i \geq \gamma_j\) for \(i \leq j\) and
\(\sum_{i=1}^{\infty} \gamma_i = 1\). At each time \(i\), let \(\tau_i\)
be the last time a discovery was made before \(i\): \[
\tau_i = \max \left\{ l \in \{1, \ldots, i-1\} : R_l = 1\right\}
\]

LORD depends on constants \(w_0\) and \(b_0\), where \(w_0 \geq 0\)
represents the initial `wealth' of the procedure and \(b_0 > 0\)
represents the `payout' for rejecting a hypothesis. We require
\(w_0+b_0 \leq \alpha\) for FDR control to hold.

Javanmard and Montanari (2018) presented three different versions of
LORD, which have different definitions of the adjusted significance
thresholds \(\alpha_i\). Versions 1 and 2 have since been superseded by
the LORD++ procedure of Ramdas \emph{et al.} (2017), so we do not
describe them here.

\begin{itemize}
\item
  \textbf{LORD++}: The significance thresholds for LORD++ are chosen as
  follows: \[
  \alpha_i = \gamma_i w_0 + (\alpha - w_0) \gamma_{i-\tau_1} + 
  \alpha \sum_{j : \tau_j < i, \tau_j \neq \tau_1} \gamma_{i - \tau_j}
  \]
\item
  \textbf{LORD 3}: The significance thresholds depend on the time of the
  last discovery time and the wealth accumulated at that time, with \[
  \alpha_i  = \gamma_{i - \tau_i} W(\tau_i)
  \] where \(\tau_1 = 0\). Here \(\{W(j)\}_{j \geq 0}\) represents the
  `wealth' available at time \(j\), and is defined recursively: \[
  \begin{align}
  W(0) & = w_0 \nonumber \\
  W(j) & = W(j-1) - \alpha_{j-1} + b_0 R_j
  \end{align}
  \]
\item
  \textbf{D-LORD}: This is equivalent to the LORD++ procedure with
  discarding. Given a discarding threshold \(\tau \in (0,1)\) and
  initial wealth \(w_0 \leq \tau\alpha\) the significance thresholds are
  chosen as follows: \[ 
  \alpha_t = \min\{\tau, \tilde{\alpha}_t\}
  \] where \[
  \tilde{\alpha}_t = w_0 \gamma_{S^t} +
  (\tau\alpha - w_0)\gamma_{S^t - \kappa_1^*} + 
  \tau\alpha \sum_{j \geq 2} \gamma_{S^t - \kappa_j^*}
  \] and \[
  \kappa_j = \min\{i \in [t-1] : \sum_{k \leq i} 
  1 \{p_k \leq \alpha_k\} \geq j\}, \;
  \kappa_j^* = \sum_{i \leq \kappa_j} 1 \{p_i \leq \tau \}, \;
  S^t = \sum_{i < t} 1 \{p_i \leq \tau \}
  \]
\end{itemize}

LORD++ is an instance of a monotone rule, and provably controls the FDR
for independent p-values provided \(w_0 \leq \alpha\). LORD 3 is a
non-monotone rule, and FDR control is only demonstrated empirically. In
some scenarios with large \(N\), LORD 3 will have a slightly higher
power than LORD++ (see Robertson \emph{et al.}, 2018), but since it is a
non-monotone rule we would recommend using LORD++ (which is the
default), especially since it also has a provable guarantee of FDR
control.

In all versions, the default sequence of \(\gamma\) is given by
\[\gamma_j = C \frac{\log(\max(j, 2))}{j e^{\sqrt{\log j}}}\] where
\(C \approx 0.07720838\), as proposed by Javanmard and Montanari (2018)
equation 31.

Javanmard and Montanari (2018) also proposed an adjusted version of LORD
that is valid for arbitrarily \emph{dependent} p-values. Similarly to
LORD 3, the adjusted significance thresholds are set equal to
\[ \alpha_i = \xi_i W(\tau_i)\] where (assuming \(w_0 \leq b_0\)),
\(\sum_{j=1}^{\infty} \xi_i (1 + \log(j)) \leq \alpha / b_0\)

The default sequence of \(\xi\) is given by
\[ \xi_j  = \frac{C \alpha }{b_0 j \log(\max(j, 2))^3}\] where
\(C \approx 0.139307\).

Note that allowing for dependent p-values can lead to a substantial loss
in power compared with the LORD procedures described above.

\hypertarget{SAFFRON}{%
\subsubsection{SAFFRON}\label{SAFFRON}}

The SAFFRON procedure controls the FDR for independent p-values, and was
proposed by Ramdas \emph{et al.} (2018). The algorithm is based on an
estimate of the proportion of true null hypotheses. More precisely,
SAFFRON sets the adjusted test levels based on an estimate of the amount
of alpha-wealth that is allocated to testing the true null hypotheses.

SAFFRON depends on constants \(w_0\) and \(\lambda\), where \(w_0\)
satisfies \(0 \leq w_0 < (1 - \lambda)\alpha\) and represents the
initial `wealth' of the procedure, and \(\lambda \in (0,1)\) represents
the threshold for a `candidate' hypothesis. A `candidate' refers to
p-values smaller than \(\lambda\), since SAFFRON will never reject a
p-value larger than \(\lambda\). These candidates can be thought of as
the hypotheses that are a-priori more likely to be non-null.

The SAFFRON procedure runs as follows:

\begin{enumerate}
\def\labelenumi{\arabic{enumi}.}
\item
  Set the initial alpha-wealth \(w_0 < (1-\lambda)\alpha\)
\item
  At each time \(i\), define the number of candidates after the \(k\)-th
  rejection as \[ C_{k+} = C_{k+}(i) = \sum_{j = \tau_k + 1}^{i-1} C_j\]
  where \(C_j = 1\{p_j \leq \lambda \}\) is the indicator for candidacy.
\item
  SAFFRON starts with \(\alpha_1 = \min\{\gamma_1 w_0, \lambda\}\).
  Subsequent test levels are chosen as
  \(\alpha_i = \min\{ \lambda, \tilde{\alpha}_i\}\), where \[
  \tilde{\alpha}_i = w_0 \gamma_{i-C_{0+}} + 
  ((1-\lambda)\alpha - w_0)\gamma_{i-\tau_1-C_{1+}} +
  (1-\lambda)\alpha \sum_{j : \tau_j < i, \tau_j \neq \tau_1} 
  \gamma_{i - \tau_j- C_{j+}}
  \]
\end{enumerate}

The default sequence of \(\gamma\) for SAFFRON is given by
\(\gamma_j \propto j^{-1.6}\).

\hypertarget{AlphaInvesting}{%
\subsubsection{Alpha-investing}\label{AlphaInvesting}}

Ramdas et al.~(2018) proposed a variant of the Alpha-investing algorithm
of Foster and Stine (2008) that guarantees FDR control for independent
p-values. This procedure uses SAFFRON's update rule with the constant
\eqn{\lambda} replaced by a sequence \(\lambda_i = \alpha_i\). This is
also equivalent to using the ADDIS algorithm (see below) with
\(\tau = 1\) and \(\lambda_i = \alpha_i\).

\hypertarget{ADDIS}{%
\subsubsection{ADDIS}\label{ADDIS}}

The ADDIS procedure controls the FDR for independent p-values, and was
proposed by Tian \& Ramdas (2019). The algorithm compensates for the
power loss of SAFFRON with conservative nulls, by including both
adapativity in the fraction of null hypotheses (like SAFFRON) and the
conservativeness of nulls (unlike SAFFRON).

ADDIS depends on constants \(w_0, \lambda\) and \(\tau\). \(w_0\)
represents the initial `wealth' of the procedure and satisfies
\(0 \leq w_0 \leq \tau \lambda \alpha\). \(\tau \in (0,1)\) represents
the threshold for a hypothesis to be selected for testing: p-values
greater than \(\tau\) are implicitly `discarded' by the procedure.
Finally, \(\lambda \in (0,1)\) sets the threshold for a p-value to be a
candidate for rejection: ADDIS will never reject a p-value larger than
\(\tau \lambda\).

The significance thresholds for ADDIS are chosen as follows: \[ 
\alpha_t = \min\{\tau\lambda, \tilde{\alpha}_t\}
\] where \[
\tilde{\alpha}_t = w_0 \gamma_{S^t-C_{0+}} +
(\tau(1-\lambda)\alpha - w_0)\gamma_{S^t - \kappa_1^*-C_{1+}} + 
\tau(1-\lambda)\alpha \sum_{j \geq 2} \gamma_{S^t - \kappa_j^* - C_{j+}}
\] and \[
\kappa_j = \min\{i \in [t-1] : \sum_{k \leq i} 
1 \{p_k \leq \alpha_k\} \geq j\}, \;
\kappa_j^* = \sum_{i \leq \kappa_j} 1 \{p_i \leq \tau \}, \;
S^t = \sum_{i < t} 1 \{p_i \leq \tau \}, \;
C_{j+} = \sum_{i = \tau_k + 1}^{t-1} 1\{p_i \leq \tau\}
\]

The default sequence of \(\gamma\) for ADDIS is the same as for SAFFRON
given \protect\hyperlink{SAFFRON_gamma}{here}.

\hypertarget{Alpha-spending}{%
\subsubsection{Alpha-spending}\label{Alpha-spending}}

The Alpha-spending procedure controls the FWER for a potentially
infinite stream of p-values using a Bonferroni-like test. Given an
overall significance level \(\alpha\), the significance thresholds are
chosen as \[\alpha_i = \alpha \gamma_i\] where
\(\sum_{i=1}^{\infty} \gamma_i = 1\) and \(\gamma_i \geq 0\). The
procedure strongly controls the FWER for arbitrarily dependent p-values.

Note that the procedure also controls the generalised familywise error
rate (k-FWER) for \(k > 1\) if \(\alpha\) is replaced by
\(\min(1,k\alpha)\).

The default sequence of \(\gamma\) is the same as that for \(\xi\) for
LORD given \protect\hyperlink{LORD_gamma}{here}.

\hypertarget{onlineFallback}{%
\subsubsection{Online Fallback}\label{onlineFallback}}

The online fallback procedure of Tian \& Ramdas (2019b) provides a
uniformly more powerful method than Alpha-spending, by saving the
significance level of a previous rejection. More specifically, online
fallback tests hypothesis \(H_i\) at level
\[\alpha_i = \alpha \gamma_i + R_{i-1} \alpha_{i-1}\] where
\(R_i = 1\{p_i \leq \alpha_i\}\) denotes a rejected hypothesis. The
procedure strongly controls the FWER for arbitrarily dependent p-values.

The default sequence of \(\gamma\) is the same as that for \(\xi\) for
LORD given \protect\hyperlink{LORD_gamma}{here}.

\hypertarget{ADDIS-spending}{%
\subsubsection{ADDIS-spending}\label{ADDIS-spending}}

The ADDIS-spending procedure strongly controls the FWER for independent
p-values, and was proposed by Tian \& Ramdas (2019b). The procedure
compensates for the power loss of Alpha-spending, by including both
adapativity in the fraction of null hypotheses and the conservativeness
of nulls.

ADDIS depends on constants \(\lambda\) and \(\tau\), where
\(\lambda < \tau\). Here \(\tau \in (0,1)\) represents the threshold for
a hypothesis to be selected for testing: p-values greater than \(\tau\)
are implicitly `discarded' by the procedure, while \(\lambda \in (0,1)\)
sets the threshold for a p-value to be a candidate for rejection:
ADDIS-spending will never reject a p-value larger than \(\lambda\).

Note that the procedure controls the generalised familywise error rate
(k-FWER) for \(k > 1\) if \(\alpha\) is replaced by \(\min(1,k\alpha)\).
Tian and Ramdas (2019b) also presented a version for handling local
dependence, see the Section on Asynchronous testing below.

The default sequence of \(\gamma\) for ADDIS-spending is the same as for
SAFFRON given \protect\hyperlink{SAFFRON_gamma}{here}.

\hypertarget{accounting-for-dependent-p-values}{%
\subsubsection{Accounting for dependent
p-values}\label{accounting-for-dependent-p-values}}

As noted above, the LORD, SAFFRON, ADDIS and ADDIS-spending procedures
assume independent p-values, while the LOND procedure is also valid
under positive dependencies (like the Benjamini-Hochberg method, see
below). Adjusted versions of LOND and LORD available for arbitrarily
dependent p-values. Alpha-spending and online fallback also control the
FWER and FDR for arbitrarily dependent p-values.

By way of comparison, the usual Benjamini-Hochberg method for
controlling the FDR assumes that the p-values are positively dependent
(PRDS). As an example, the PRDS is satisfied for multivariate normal
test statistics with a positive correlation matrix). See Benjamini \&
Yekutieli (2001) for further technical details.

\begin{center}\rule{0.5\linewidth}{0.5pt}\end{center}

\hypertarget{asynchronous-testing}{%
\subsection{Asynchronous testing}\label{asynchronous-testing}}

Zrnic \emph{et al.} (2018) proposed procedures to control the modified
FDR (mFDR) in the context of \emph{asynchronous} testing, i.e.~where
each hypothesis test can itself be a sequential process and the tests
can overlap in time. They presented asynchronous versions of the LOND,
LORD and SAFFRON procedures for a variety of trial settings, including
the following:

1: \textbf{Asynchronous online mFDR control}: This is for an
asynchronous testing process, consisting of tests that start and finish
at (potentially) random times. The discretised finish times of the test
correspond to the decision times.

2: \textbf{Online mFDR control under local dependence}: For any \(t>0\)
we allow the p-value \(p_t\) to have arbitrary dependence on the
previous \(L_t\) p-values. The fixed sequence \(L_t\) is referred to as
`lags'.

3: \textbf{mFDR control in asynchronous mini-batch testing}: A
mini-batch represents a grouping of tests run asynchronously which
result in dependent p-values. Once a mini-batch of tests is fully
completed, a new one can start, testing hypotheses independent of the
previous batch.

\begin{center}\rule{0.5\linewidth}{0.5pt}\end{center}

\hypertarget{variations-to-the-default-options}{%
\subsection{Variations to the default
options}\label{variations-to-the-default-options}}

In the following section, we consider the arguments that a typical user
might consider amending for their analysis.

\hypertarget{common-arguments}{%
\subsubsection{Common arguments}\label{common-arguments}}

As a default, the \texttt{alpha} argument is set to 0.05, where
\texttt{alpha} sets the overall significance level of the FDR of FWER
controlling procedure. By convention, the standard significance level
utilised is the 5\%. However, there are applications where an alternate
threshold could be considered. For example, a more stringent threshold
might be appropriate when there are limited resources to follow up
significant findings. A less stringent threshold might be appropriate
when the downstream analysis is a global analysis which can tolerate a
higher proportion of false positives.

To ensure correct interpretation of the dates provided there is a
date.format argument. As a default, the date format is set to receive
dates as year-month(00-12)-day(number). The following website provides
clear guidance on symbols used to interpret the date information:
\url{https://www.statmethods.net/input/dates.html}

As a default, the \texttt{random} argument is set to \texttt{TRUE}. In
this situation, the order of p-values in each batch (i.e.~with the same
date) are randomised. This is to avoid the risk of p-values being
ordered post-hoc, which can lead to an inflation of the FDR. As the
dataset grows the data is reprocessed. To ensure the consistency of the
output (with the randomisation within the previous batches remaining the
same), it is necessary to set the same \texttt{seed} for all analyses.

The user also has the option to turn off the randomisation step, by
setting the \texttt{random} argument to \texttt{FALSE}. This approach
would be appropriate if the user has both a date \emph{and} a time stamp
for the p-values, in which case the data should be ordered by date and
time beforehand and then passed to a wrapper function. Another scenario
would be when p-values within the batches are ordered using
\emph{independent} side information, so that hypotheses most likely to
be rejected come first, which would potentially increase the power of
the procedure (see Javanmard and Montanari (2018) and Li and Barber
(2017)).

\hypertarget{lond}{%
\subsubsection{LOND}\label{lond}}

As a default, the \texttt{dep} argument is set to \texttt{FALSE}.
Alternatively, this can be set to \texttt{TRUE} and will implement the
LOND procedure to guarantee FDR control for arbitrarily dependent
p-values. This method will in general be more conservative.

\begin{Shaded}
\begin{Highlighting}[]
\KeywordTok{set.seed}\NormalTok{(}\DecValTok{1}\NormalTok{); results.indep <-}\StringTok{ }\KeywordTok{LOND}\NormalTok{(sample.df)    }\CommentTok{# for independent p-values}
\KeywordTok{set.seed}\NormalTok{(}\DecValTok{1}\NormalTok{); results.dep <-}\StringTok{ }\KeywordTok{LOND}\NormalTok{(sample.df, }\DataTypeTok{dep=}\OtherTok{TRUE}\NormalTok{)   }\CommentTok{# for dependent p-values}

\CommentTok{# compare adjusted significance thresholds}
\KeywordTok{cbind}\NormalTok{(}\DataTypeTok{independent =}\NormalTok{ results.indep}\OperatorTok{$}\NormalTok{alphai, }\DataTypeTok{dependent =}\NormalTok{ results.dep}\OperatorTok{$}\NormalTok{alphai)}
\CommentTok{#>        independent    dependent}
\CommentTok{#>  [1,] 0.0026758385 0.0026758385}
\CommentTok{#>  [2,] 0.0011638206 0.0007758804}
\CommentTok{#>  [3,] 0.0009912499 0.0005406818}
\CommentTok{#>  [4,] 0.0008243606 0.0003956931}
\CommentTok{#>  [5,] 0.0006988870 0.0003060819}
\CommentTok{#>  [6,] 0.0006045900 0.0002467714}
\CommentTok{#>  [7,] 0.0005319444 0.0002051576}
\CommentTok{#>  [8,] 0.0007117838 0.0002618915}
\CommentTok{#>  [9,] 0.0006421423 0.0002269882}
\CommentTok{#> [10,] 0.0007796504 0.0002661860}
\CommentTok{#> [11,] 0.0007155186 0.0002369363}
\CommentTok{#> [12,] 0.0006610273 0.0002130140}
\CommentTok{#> [13,] 0.0006141682 0.0001931265}
\CommentTok{#> [14,] 0.0005734509 0.0001763616}
\CommentTok{#> [15,] 0.0005377472 0.0001620585}
\end{Highlighting}
\end{Shaded}

The vector \texttt{betai} is supplied by default, but can optionally be
specified by the user (as described above, see the formula for
\(\beta_j\) \protect\hyperlink{LOND_beta}{here}).

\hypertarget{lord}{%
\subsubsection{LORD}\label{lord}}

The default version of LORD used is version `++', but the user can
optionally specify versions 3, `discard' and `dep' using the
\texttt{version} argument (see \protect\hyperlink{LORD}{here} for
further details about the different versions).

\begin{Shaded}
\begin{Highlighting}[]
\KeywordTok{set.seed}\NormalTok{(}\DecValTok{1}\NormalTok{); results.LORD.plus <-}\StringTok{ }\KeywordTok{LORD}\NormalTok{(sample.df)}
\KeywordTok{set.seed}\NormalTok{(}\DecValTok{1}\NormalTok{); results.LORD3 <-}\StringTok{ }\KeywordTok{LORD}\NormalTok{(sample.df, }\DataTypeTok{version=}\DecValTok{3}\NormalTok{)}
\KeywordTok{set.seed}\NormalTok{(}\DecValTok{1}\NormalTok{); results.LORD.discard <-}\StringTok{ }\KeywordTok{LORD}\NormalTok{(sample.df, }\DataTypeTok{version=}\StringTok{'discard'}\NormalTok{)}
\KeywordTok{set.seed}\NormalTok{(}\DecValTok{1}\NormalTok{); results.LORD.dep <-}\StringTok{ }\KeywordTok{LORD}\NormalTok{(sample.df, }\DataTypeTok{version=}\StringTok{'dep'}\NormalTok{) }

\CommentTok{# compare adjusted significance thresholds}
\KeywordTok{cbind}\NormalTok{(}\DataTypeTok{LORD.plus =}\NormalTok{ results.LORD.plus}\OperatorTok{$}\NormalTok{alphai,}
    \DataTypeTok{LORD3 =}\NormalTok{ results.LORD3}\OperatorTok{$}\NormalTok{alphai,}
    \DataTypeTok{LORD.discard  =}\NormalTok{ results.LORD.discard}\OperatorTok{$}\NormalTok{alphai,}
    \DataTypeTok{LORD.dep =}\NormalTok{ results.LORD.dep}\OperatorTok{$}\NormalTok{alphai)}
\CommentTok{#>          LORD.plus        LORD3 LORD.discard     LORD.dep}
\CommentTok{#>  [1,] 0.0002675839 0.0002675839 0.0002675839 2.323935e-03}
\CommentTok{#>  [2,] 0.0024664457 0.0026615183 0.0011285264 1.107961e-02}
\CommentTok{#>  [3,] 0.0005732818 0.0005787961 0.0002823266 1.855138e-03}
\CommentTok{#>  [4,] 0.0004872805 0.0004929725 0.0002394680 6.924756e-04}
\CommentTok{#>  [5,] 0.0004059066 0.0004099744 0.0001998165 3.540284e-04}
\CommentTok{#>  [6,] 0.0003447286 0.0003475734 0.0001700069 2.138161e-04}
\CommentTok{#>  [7,] 0.0002986627 0.0003006772 0.0001475152 1.430752e-04}
\CommentTok{#>  [8,] 0.0029389397 0.0048133468 0.0014680343 1.685669e-04}
\CommentTok{#>  [9,] 0.0008168502 0.0010467508 0.0014680343 1.270096e-04}
\CommentTok{#> [10,] 0.0033835974 0.0069079880 0.0017451837 1.560048e-04}
\CommentTok{#> [11,] 0.0011873999 0.0015022690 0.0006438778 1.255746e-04}
\CommentTok{#> [12,] 0.0010225858 0.0012795133 0.0006438778 1.034364e-04}
\CommentTok{#> [13,] 0.0008785607 0.0010640913 0.0006438778 8.681710e-05}
\CommentTok{#> [14,] 0.0007679398 0.0009021289 0.0005497556 7.401343e-05}
\CommentTok{#> [15,] 0.0006820264 0.0007804097 0.0005497556 6.393279e-05}
\end{Highlighting}
\end{Shaded}

By default \(w_0 = \alpha/10\) and (for LORD 3 and LORD dep)
\(b0 = alpha - w0\), but these parameters can optionally be specified by
the user subject to the requirements that \(0 \leq w_0 \leq \alpha\),
\(b_0 > 0\) and \(w_0+b_0 \leq \alpha\).

The value of \texttt{gammai} is also supplied by default, but can
optionally be specified by the user (as described above, see the formula
for \(\gamma_j\) \protect\hyperlink{LORDdep_xi}{here} for version=`dep'
and \protect\hyperlink{LORD_gamma}{here} for all other versions of
LORD).

\hypertarget{saffron}{%
\subsubsection{SAFFRON}\label{saffron}}

By default \(w_0 = \alpha/2\) and \(\lambda = 0.5\), but these
parameters can optionally be specified by the user subject to the
requirements that \(0 \leq w_0 < \alpha\) and \(0 < \lambda < 1\). The
values of \texttt{gammai} are also supplied by default, but can
optionally be specified by the user (as described above, see the formula
for \(\gamma_j\) \protect\hyperlink{SAFFRON_gamma}{here}).

\hypertarget{addis}{%
\subsubsection{ADDIS}\label{addis}}

By default \(w_0 = \tau \lambda \alpha/2\) and \(\lambda = \tau = 0.5\),
but these parameters can optionally be specified by the user subject to
the requirements that \(0 \leq w_0 < \tau \lambda \alpha\),
\(0 < \lambda < 1\) and \(0 < \tau < 1\). The values of \texttt{gammai}
are also supplied by default, but can optionally be specified by the
user.

\hypertarget{alpha-spending-and-online-fallback}{%
\subsubsection{Alpha-spending and online
fallback}\label{alpha-spending-and-online-fallback}}

The values of \texttt{gammai} are supplied by default, but can
optionally be specified by the user.

\hypertarget{addis-spending}{%
\subsubsection{ADDIS-spending}\label{addis-spending}}

By default \(\lambda = 0.25\) and \(\tau = 0.5\), but these parameters
can optionally be specified by the user subject to the requirements that
\(\lambda < \tau\), \(0 < \lambda < 1\) and \(0 < \tau < 1\). The values
of \texttt{gammai} are also supplied by default, but can optionally be
specified by the user.

\begin{center}\rule{0.5\linewidth}{0.5pt}\end{center}

\hypertarget{acknowledgements}{%
\subsection{Acknowledgements}\label{acknowledgements}}

We would like to thank the IMPC team (via Jeremy Mason and Hamed Haseli
Mashhadi) for useful discussions during the development of the package.

\begin{center}\rule{0.5\linewidth}{0.5pt}\end{center}

\hypertarget{references}{%
\subsection{References}\label{references}}

Aharoni, E. and Rosset, S. (2014). Generalized \(\alpha\)-investing:
definitions, optimality results and applications to public databases.
\emph{Journal of the Royal Statistical Society (Series B)},
76(4):771--794.

Benjamini, Y., and Yekutieli, D. (2001). The control of the false
discovery rate in multiple testing under dependency. \emph{The Annals of
Statistics}, 29(4):1165-1188.

Bourgon, R., Gentleman, R., and Huber, W. (2010). Independent filtering
increases detection power for high-throughput experiments.
\emph{Proceedings of the National Academy of Sciences}, 107(21),
9546-9551.

Foster, D. and Stine R. (2008). \(\alpha\)-investing: a procedure for
sequential control of expected false discoveries. \emph{Journal of the
Royal Statistical Society (Series B)}, 29(4):429-444.

Ioannidis, J.P.A. (2005). Why most published research findings are
false. \emph{PLoS Medicine}, 2.8:e124.

Javanmard, A., and Montanari, A. (2015). On Online Control of False
Discovery Rate. \emph{arXiv preprint},
\url{https://arxiv.org/abs/1502.06197}.

Javanmard, A., and Montanari, A. (2018). Online Rules for Control of
False Discovery Rate and False Discovery Exceedance. \emph{Annals of
Statistics}, 46(2):526-554.

Koscielny, G., \emph{et al}. (2013). The International Mouse Phenotyping
Consortium Web Portal, a unified point of access for knockout mice and
related phenotyping data. \emph{Nucleic Acids Research},
42.D1:D802-D809.

Li, A., and Barber, F.G. (2017). Accumulation Tests for FDR Control in
Ordered Hypothesis Testing. \emph{Journal of the American Statistical
Association}, 112(518):837-849.

Ramdas, A., Yang, F., Wainwright M.J. and Jordan, M.I. (2017). Online
control of the false discovery rate with decaying memory. \emph{Advances
in Neural Information Processing Systems 30}, 5650-5659.

Ramdas, A., Zrnic, T., Wainwright M.J. and Jordan, M.I. (2018). SAFFRON:
an adaptive algorithm for online control of the false discovery rate.
\emph{Proceedings of the 35th International Conference in Machine
Learning}, 80:4286-4294.

Robertson, D.S. and Wason, J.M.S. (2018). Online control of the false
discovery rate in biomedical research. \emph{arXiv preprint},
\url{https://arxiv.org/abs/1809.07292}.

Robertson, D.S., Wildenhain, J., Javanmard, A. and Karp, N.A. (2019).
Online control of the false discovery rate in biomedical research.
\emph{Bioinformatics}, 35:4196-4199,
\url{https://doi.org/10.1093/bioinformatics/btz191}.

Tian, J. and Ramdas, A. (2019a). ADDIS: an adaptive discarding algorithm
for online FDR control with conservative nulls. \emph{arXiv preprint},
\url{https://arxiv.org/abs/1905.11465}.

Tian, J. and Ramdas, A. (2019b). Online control of the familywise error
rate. \emph{arXiv preprint}, \url{https://arxiv.org/abs/1910.04900}.

Zrnic, T., Ramdas, A. and Jordan, M.I. (2018). Asynchronous Online
Testing of Multiple Hypotheses. \emph{arXiv preprint},
\url{https://arxiv.org/abs/1812.05068}.

\end{document}
